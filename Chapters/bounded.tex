
\setcounter{section}{5}
\section{Bounded Operators}

\subsection{Topologies on bounded operators}

one Banach space to another. In this chapter we will study $\Lin{X,Y}$ more
closely. We emphasize the case which will arise most frequently later, namely,
$\Lin{\H, \H} \equiv \Lin{H}$ where $\H$ is a separable Hilbert space. Theorem III.2
shows that $\Lin{X,Y}$ is a Banach space with the norm \begin{align}
    \norm{\op T} = \sup_{x \neq 0} \frac{\norm{\op T x}_Y}{\norm{x}_X} \:.
\end{align}
The induced topology on $\Lin{X,Y}$ is called the \textbf{uniform operator topology} (or \textbf{norm topology}). In this topology the map $[A,B] \mapsto BA$ of $\Lin{X,Y} \times \Lin{X,Y} \to \Lin{X,Z}$ is jointly continuous.
We now introduce two new topologies on $\Lin{X,Y}$, the weak and strong operator topologies. There are other interesting and useful topologies on $\Lin{X,Y}$, but we delay their introduction until we need them in a later volume (see however the discussion at the end of Section 6 and the Notes).

The \textbf{strong operator topology} is the weakest topology on $\Lin{X,Y}$ such that the maps \begin{align}
    E_x : \Lin{X,Y} \to Y 
\end{align}
given by $E_X(\op T) = \op T x$ are continuous for all $x \in X$. A neighborhood basis at the origin is given by sets of the form \begin{align}
    \set{\op S | \op S \in \Lin{X,Y} \:, \: \norm{\op S x_i}_Y < \varepsilon \:, \: i = 1, \cdots, n}
\end{align}
where $\set{x_i}_{i=1}^n$ is a finite collection of elements of $X$ and $\varepsilon$ is positive. In this topology a net $\set{T_\alpha}$ of operators converges to an operator $\op T$ (written $\op T_\alpha \cs \op T$ if and only if $\norm{\op T_\alpha x - \op T x}$ for all $x \in X$. The map $[\op A, \op B] \mapsto \op A \op B$ is separately but not jointly continuous if $X$, $Y$, and $Z$ are infinite dimensional (see Problem 6a, b). We sometimes denote strong limits by the symbol $\slim$.

The \textbf{weak operator topology} on $\Lin{X,Y}$ is the weakest topology such that the maps
\begin{align}
    E_{x, \ell} : \Lin{X,Y} \to \C 
\end{align}
given by $E_{x, \ell}(\op T) = \ell(\op T x)$ are all continuous for all $x \in X$, $\ell \in Y^\star$. A basis at the origin is given by sets of the form \begin{align}
    \set{\op S | \op S \in \Lin{X,Y} \:; \: |\ell_i(\op T x_j)| < \varepsilon \:; \: i = 1, \cdots , n \:; \: j = 1, \cdots, m} 
\end{align}
where $\set{x_i}_{i=1}^n$ and $\set{\ell_j}_{j=1}^m$ are finite families of elements of $X$ and $Y^\star$ respectively. A net of operators $\set{\op T_\alpha}$ converges to an operator $\op T$ in the weak operator topology (written $\op T_\alpha \cw \op T$) if and only if $|\ell (\op T_\alpha x) - \ell (\op T x)| \rightarrow 0$ for each $l \in Y^\star$ and $x \in X$. Notice that in the case $\Lin{\H}$, $T_\gamma \cw$ just means that the \enquote{matrix elements} $(y, \op T_\gamma x)$ converge to $(y, \op T x)$. In the weak topology the map $[\op A, \op B] \mapsto \op A \op B$ is separately, but not jointly continuous if $X$, $Y$ and $Z$ are infinite dimensional (see Problem 6c).

\begin{remark}
    The reader should not confuse the weak operator topology on $\Lin{X,Y}$ with the weak (Banach space) topology on $\Lin{X,Y}$. The former is the weakest topology such that the bounded linear functionals on $\Lin{X,Y}$ of the form $\ell(\cdot x)$ are continuous for all $x \in X$ and $\ell \in Y^\star$. The latter is the weakest topology such that \textit{all} bounded linear functionals on $\Lin{X,Y}$ are continuous (see Section VI.6).
\end{remark}

Notice that the weak operator topology is weaker than the strong operator topology which is weaker than the uniform operator topology. In general, the weak and strong operator topologies on $\Lin{X,Y}$ will not be first countable so that questions of compactness, net convergence, and sequential convergence are complicated. The following simple example illustrates the different topologies on $\Lin{\ell_2}$.

\begin{example}
    Consider the bounded operators on $\ell_2$.
    \begin{enumerate}
        \item Let $\op T_n$ be defined by \begin{align}
            \op T_n (\xi_1, \xi_2, \cdots) = \left( \frac{1}{n} \xi_1, \frac{1}{n} \xi_2, \cdots \right) \:.
        \end{align}
        Then $\op T_n \rightarrow 0$ uniformly.

        \item Let $\op S_n$ be defined by \begin{align}
            \op S_n(\xi_1, \xi_2, \cdots) = \left( 0,0, \cdots, 0, \xi_{n+1}, \xi_{n+2}, \cdots \right) \:.
        \end{align}
        Then $\op S_n \rightarrow 0$ strongly but not uniformly.

        \item Let $\op W_n$ be defined by \begin{align}
            \op W_n(\xi_1, \xi_2, \cdots) = \left( 0,0, \cdots, 0, \xi_1, \xi_2, \cdots \right) \:.
        \end{align}
        Then $\op W_n \rightarrow 0$ in the weak operator topology but not in the strong or uniform topologies.
    \end{enumerate}

\end{example}

The following result in the Hilbert space case is sometimes useful and provides
a nice application of the uniform boundedness theorem.

\begin{theorem}
    Let $\Lin{H}$ denote the bounded operators on a Hilbert space $\H$. Let $\op T_n$ be a \textit{sequence} of bounded operators and suppose that $(\op T_n x, y)$ converges as $n \rightarrow \infty$ for each $x,y \in \H$. Then there exists $\op T \in \Lin{\H}$ such that $\op T_n \cw \op T$.
\end{theorem}

\begin{proof}
    We begin by showing that for each $x$, $\sup_n \norm{\op T_n x} < \infty$. Since for any $x \in \H$, $(x, \op T_n y)$ converges we have \begin{align}
        \sup_n |(\op T_n x, y)| < \infty \:.
    \end{align}
    For each $n$, $\op T_n x \in \Lin{\H, \C}$, and since $\sup_n |(\op T_n x)(y)|_{\C} < \infty$, the uniform boundedness theorem implies that the operator norms of the $\op T_nx$ in $\Lin{\H, \C}$ is the same as its norm in $\H$; thus $\norm{\op T_n x}_{\H}$ is uniformly bounded.

    Now, we use the uniform boundedness theorem again. Since \begin{align}
        \sup_n \norm{\op T_n x}_{\H} < \infty \:,
    \end{align}
    we conclude \begin{align}
        \sup_n \norm{\op T_n}_{\Lin{\H}} < \infty \:.
    \end{align}
    Define $B(x,y) = \lim_n (\op T_n x, y)$. Then it is easily verified that $B(x,y)$ is sesquilinear and \begin{align}
        |B(x,y)| \leq \lim_n |(\op T_n x,y)| \leq \norm{x} \norm{y} (\sup_n \norm{\op T_n}) \:.
    \end{align}
    Thus $B(x,y)$ is a bounded sesquilinear form on $\H$ and so, by the corollary to the Riesz lemma, there is a bounded operator $\op T \in \Lin{\H}$ sush that $B(x,y) = (\op T x,y)$. Clearly $\op T_n \cw \op T$.
\end{proof}

If a sequence of operators $\op T_n$, on a Hilbert space has the property that $\op T_n x$ converges for each $x \in \H$, then there exists $\op T \in \Lin{\H}$ such that $\op T_n \cs \op T$. The reader is asked to prove this theorem and various generalizations in Problem 3.

Let $\op T \in \Lin{X,Y}$. The set of vectors $x \in X$ so that $\op T x = 0$ is called the \textbf{kernel} of $\op T$, written $\Ker \op T$. The set of vectors $y \in Y$ that $y = \op T x$ for some $x \in X$ is called the \textbf{range} of $\op T$, written $ \Ran \op T$. Notice that both $\Ker \op T$ and $\Ran \op T$ are subspaces. Ker $\op T$ is necessarily closed, but $\Ran \op T$ may not be closed (Problem 7).

\subsection{Adjoints}

In this section we define adjoints of bounded operators on Banach and Hilbert spaces. The reader should be cautioned at the outset that the Hilbert space adjoint of an operator $\op T \in \Lin{H}$ is not equal to the Banach space adjoint although it is closely related to it.

\begin{definition}
    Let $X$ and $Y$ be Banach spaces, $\op T$ a bounded linear operator from $X$ to $Y$. The Banach space \textbf{adjoint} of $\op T$, denoted by $\op T'$, is the bounded linear operator from $Y^\star$ to $X^\star$ defined by
    \begin{align}
        (\op T' \ell)(x) = \ell(\op T x) \quad \text{ for all } \quad \ell \in Y^\star \:, \: x \in X \:.
    \end{align}
\end{definition}

\begin{example}
    Let $X = \ell_1 = Y$ and let $\op T$ be the right shift operator
    \begin{align}
        \op T (\xi_1, \xi_2, \cdots) = (0, \xi_1, \xi_2, \cdots) \:.
    \end{align}
    Then $\op T' : \ell_\infty \to \ell_\infty$ is the operator
    \begin{align}
        \op T'(\xi_1, \xi_2, \cdots) = (\xi_2, \xi_3, \cdots) \:.
    \end{align}
    In this example, $\norm{\op T} = 1 = \norm{\op T'}$. In fact the norms of $\op T$ and $\op T'$ are always equal:
\end{example}

\begin{theorem}
    Let $X$ and $Y$ be Banach spaces. The map $\op T \mapsto \op T'$ is an isometric isomorphism of $\Lin{X,Y}$ into $\Lin{X^\star,Y^\star}$.
\end{theorem}

\begin{proof}
    The map $\op T \mapsto \op T'$ is linear. The fact that $\op T'$ is bounded and that the map is an isometry follows from the computation ($\ell \in Y^\star$)
    \begin{align}
        \norm{\op T}_{\Lin{X,Y}} = \sup_{\norm{x} \leq 1} \norm{\op T x}_Y
        = \sup_{\norm{\ell} \leq 1} \left( \sup_{\norm{x} \leq 1} |\ell(\op T x)|\right)
        = \sup_{\norm{\ell} \leq 1} \norm{\op T' \ell} = \norm{\op T'}_{\Lin{Y^\star, X^\star}} \:.
    \end{align}
    The second equality uses a corollary of the Hahn Banach theorem.
\end{proof}

We are mostly interested in the case where $\op T$ is a bounded linear transformation of a Hilbert space $\H$ to itself. The Banach space adjoint of $\op T$ is then a mapping of $\H^\star$ to $\H^\star$. Let $\op C : \H \to \H^\star$ be the map which assigns to each $y \in \H$, the bounded linear functional $(y, \cdot)$ in $\H^\star$. $\op C$ is a \textit{conjugate} linear isometry which is surjective by the Riesz lemma. Now define a map $\op T^\star : \H \to \H$ by \begin{align}
    \op T^\star = \op C^{-1} \op T' \op C \:.
\end{align}
Then $\op T^\star$ satisfies \begin{align}
    (x, \op T y) = (\op C x)(\op T y)= (\op T' \op C x)(y) = (\op C^{-1} \op T' \op C x, y) = (\op T^\star x,y) \:.
\end{align}

\begin{theorem}
    \begin{enumerate}
        \item $\op T \mapsto \op T^\star$ is a conjugate linear isometric isomorphism of $\Lin{\H}$ onto $\Lin{\H}$.
        \item $(\op T \op S)^\star = \op S^\star \op T^\star$.
        \item $(\op T^\star)^\star = \op T$.
        \item If $\op T$ has a bounded inverse, $\op T^{-1}$, then $\op T^\star$ has a bounded inverse and $(\op T^\star)^{-1} = (\op T^{-1})^\star$.
        \item The map $\op T \mapsto \op T^\star$ is always continuous in the weak and uniform operator topologies but is only continuous in the strong operator topology if $\H$ is finite dimensional.
        \item $\norm{\op T^\star \op T} = \norm{\op T}^2$.
    \end{enumerate}
\end{theorem}

\begin{proof}
    (1) follows from Theorem VI.2 and the fact that $\op C$ is an isometry. (2) and (3) are easily checked. Since $\op T^{-1} \op T = \op I = \op T \op T^{-1}$, we have from (2)
    \begin{align}
        \op T^\star (\op T^{-1})^\star = \op I^\star = \op I = \op I^\star = (\op T^{-1})^\star \op T^\star
    \end{align}
    which proves (4).
    TODO (5)
\end{proof}

\begin{definition}
    A bounded operator $\op T$ on a Hilbert space is called \textbf{self-adjoint} if $\op T = \op T^\star$.
\end{definition}

Self-adjoint operators play a major role in functional analysis and mathematical physics and much of our time is devoted to studying them. Chapter VII is devoted to proving a structure theorem for bounded self-adjoint operators. In Chapter VIII we introduce unbounded self-adjoint operators and continue their study in Chapter X. We remind the reader that on $\C^n$, a linear transformation is self-adjoint if and only if its matrix in any orthonormal basis is invariant under the operation of reflection across the diagonal followed by complex conjugation.

An important class of operators on Hilbert spaces is that of the projections.

\begin{definition}
    If $\op P \in \Lin{\H}$ and $\op P^2 = \op P$, then $\op P$ is called a \textbf{projection}. If in addition $\op P = \op P^\star$, then $\op P$ is called an \textbf{orthogonal projection}.
\end{definition}

Notice that the range of a projection is always a closed subspace on which $\op P$ acts like the identity. If in addition $\op P$ is orthogonal, then $\op P$ acts like the zero operator on $(\Ran \op P)^\bot$. If $x = y + z$, with $y \in \Ran \op P$ and $z \in (\Ran \op P)^\bot$, is the decomposition guaranteed by the projection theorem, then $\op P x =y$. $\op P$ is called the orthogonal projection onto $\Ran \op P$. Thus, the projection theorem sets up a one to one correspondence between orthogonal projections and closed subspaces. Since orthogonal projections arise more frequently than nonorthogonal ones, we normally use the word projection to mean orthogonal projection.

\subsection{The spectrum}

If $\op T$ is a linear transformation on $\C^n$, then the eigenvalues of $\op T$ are the complex numbers $\lambda$ 
such that the determinant of $( \lambda \op J - \op T )$ is equal to zero. 
The set of such $\lambda$ is called the spectrum of $\op T$. It can consist of at most $n$ points since $\det (\lambda \op I - \op T)$ is a polynomial of degree $n$. If $\lambda$ is not an eigenvalue, then $\lambda \op I - \op T$ has an inverse since $\det (\lambda \op I - \op T) \neq 0$.
The spectral theory of operators on infinite-dimensional spaces is more complicated, more interesting, and very important for an understanding of the operators themselves.

\begin{definition}
    Let $\op T \in \Lin{X}$. A complex number $\lambda$ is said to be in the \textbf{resolvent set} $\rho(\op T)$ of $\op T$ if $\lambda \op I - \op T$ is a bijection with a bounded inverse. $\op R_{\lambda}(\op T) = (\lambda \op I - \op T)^{-1}$ is called the \textbf{resolvent} of $\op T$ at $\lambda$. If $\lambda \neq \rho(\op T)$, then $\lambda$ is said to be in the \textbf{spectrum} $\sigma(\op T)$ of $\op T$.
\end{definition}

We note that by the inverse mapping theorem, $\lambda \op I - \op T$ automatically has a bounded inverse if it is bijective. We distinguish two subsets of the spectrum.

\begin{definition}
    Let $\op T \in \Lin{H}$.
    \begin{enumerate}
        \item An $x \neq 0$ which satisfies $\op T x = \lambda x$ for some $\lambda \in \C$ is called an \textbf{eigenvector} of $\op T$; $\lambda$ is called the corresponding \textbf{eigenvalue}. If $\lambda$ is an eigenvalue, then $\lambda \op I - \op T$ is not injective so $\lambda$ is in the spectrum of $\op T$. The set of all eigenvalues is called the \textbf{point spectrum} of $\op T$.
        \item If $\lambda$ is not an eigenvalue and if $\Ran (\lambda \op I - \op T)$ is not dense, then $\lambda$ is said to be in the \textbf{residual spectrum}.
    \end{enumerate}
\end{definition}

At the end of this section we present an example which illustrates these kinds of spectra. The reason that we single out the residual spectrum is that it does not occur for a large class of operators, for example, for self-adjoint operators (see Theorem VI.8).

The spectral analysis of operators is very important for mathematical physics. For example, in quantum mechanics the Hamiltonian is an unbounded self-adjoint operator on a Hilbert space. The point spectrum of the Hamil- tonian corresponds to the energy levels of bound states of the system. The rest of the spectrum plays an important role in the scattering theory of the system (see Chapter XII).

We will shortly prove that the resolvent set $\rho(\op T)$ is open and that $\op R_{\lambda}(\op T)$ is an analytic operator-valued function on $\rho(\op T)$. This fact allows one to use complex analysis to study $\op R_{\lambda}(\op T)$ and thus to obtain information about $\op T$. We begin
with a brief aside about vector-valued analytic functions.

Let $X$ be a Banach space and let $D$ be a region in the complex plane, i.e. a connected open subset of $\C$. A function, $x(\cdot)$, defined on $D$ with values in $X$, is said to be \textbf{strongly analytic} at $z_0 \in D$ if the limit of $\frac{x(z_0+h)-x(z_0)}{h}$ exists in $X$ as $h$ goes to zero in $\C$. Starting from this point one can develop a theory of vector-valued analytic functions which is almost exactly parallel to the usual theory; in particular, a strongly analytic function has a norm-convergent Taylor series. We do not repeat this development here; see the notes for references. We do want to discuss one important point. There is another natural way to define Banach-valued analytic functions. Namely: a
function $x(\cdot)$ on $D$ with values in $X$ is said to be \textbf{weakly analytic} if $\ell(x(\cdot))$ is a complex valued analytic function on $D$ for each $\ell \in X^\star$. Although this second definition of analytic is a priori weaker than the first, the two definitions are equivalent, a fact we will prove in a moment. This is very important, since weak analyticity is often much easier to check.

\begin{lemma}
    Let $X$ be a Banach space. Then a sequence $\set{x_n}$ is Cauchy if and only if $\set{\ell(x_n)}$ is Cauchy, uniformly for $\ell \in X^\star$, $\norm{\ell} \leq 1$.
\end{lemma}

\begin{proof}
    If $\set{x_n}$ is Cauchy, then $|\ell(x_n) - \ell(x_m)| \leq \norm{x_n-x_m}$ for all $\ell$ with $\norm{\ell} \leq 1$, so $\set{\ell(x_n)}$ is Cauchy uniformly. Conversely,
    \begin{align}
        \norm{x_n-x_m} = \sup_{\norm{\ell} \leq 1} |\ell(x_n-x_m)| \:.
    \end{align} 
    Thus, if $\set{\ell(x_n)}$ is Cauchy, uniformly for $\norm{\ell} \leq 1$, then $\set{x_n}$ is norm-Cauchy.
\end{proof}

\begin{theorem}
    Every weakly analytic function is strongly analytic.
\end{theorem}

\begin{proof}
    Let $x(\cdot)$ be a weakly analytic function on $D$ with values in $X$.Let $z_0 \in D$ and suppose $\Gamma$ is a circle in $D$ containing $z_0$ whose interior is contained in $D$. If $\ell \in X^\star$ then $\ell(x(z))$ is analytic and
    \begin{align}
        \ell \left( \frac{x(z_0+h)-x(z_0)}{h}\right) - \der{}{z} \ell(x(z_0)) = \\
        = \frac{1}{2 \pi \i} \oint_\Gamma \left[ \frac{1}{h} \left( \frac{1}{z-(z_0+h)} - \frac{1}{z-z_0} \right) - \frac{1}{(z-z_0)^2} \right] \ell (x(z)) \, \D z \:.
    \end{align}
    Since $\ell(x(z))$ is continuous on $\Gamma$ and $\Gamma$ is compact, $|\ell(x(z)) \leq C_\ell$ for all $z \in \Gamma$. Regarding $x(z)$ as a family of mappings $x(z) : X^\star \to \C$ we see that $x(z)$ is pointwise bounded at each $\ell$ so by the uniform boundedness theorem $\sup_{z \in \Gamma} \norm{x(z)} \leq C < \infty$. Thus \begin{align}
        \left| \ell \left( \frac{x(z_0+h)-x(z_0)}{h}\right) - \der{}{z} \ell(x(z_0)) \right| \leq \\
        \leq \frac{1}{2 \pi} \norm{\ell} \left( \sup_{z \in \Gamma} \norm{x(z)} \right) \oint_\Gamma \left| \frac{1}{(z-(z_0+h))(z-z_0)} - \frac{1}{(z-z_0)^2}\right| \, \D z
    \end{align}

\end{proof}
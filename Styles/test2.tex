\section{Hlavní část}

\subsection{Vedlejší nadpis}

V této kapitole se budeme zabývat spojitými zobrazeními. Zobrazení $f: \R \to \R$ nazýváme \textbf{spojité}, jestliže platí \begin{align}
    \forall \delta >0 \, \exists \varepsilon > 0 : \, |x-x_0| < \delta \implies |f(x) - f(x_0)| < \varepsilon \label{eq:1-spojitost}
\end{align}

Od této chvíle budeme značit vektorové funkce $\vc F$ a matice $\mat M$. Například máme lineární zobrazení $\vc F: \R^n \to \C^m$ definované předpisem \begin{align}
    \vc F x := \mat M x = \sum_{i=1}^n M_{ij} x_i \label{eq:2} \:,
\end{align}
to je jistě spojité zobrazení, neboť vyhovuje podmínce \eqref{eq:1-spojitost}.